\documentclass[tikz,border=10pt,multi,rgb]{standalone}
\usepackage[edges]{forest}
\usepackage{tikz}
\usepackage{textcomp,fixltx2e}
\usepackage{alphalph}
\usepackage{forarray}
\usetikzlibrary{fit,calc,shapes,positioning,scopes,backgrounds,arrows,arrows.meta}
\begin{document}

\forestset{
  tree node/.style = {
%	  inner sep=2pt,
%	  rounded corners = 2pt,
%	  font=\sffamily,
	  rectangle,
	  draw, 
%	  thick, 
%	  fill=white, 
%	  drop shadow
	}
}

\def\toplevel{}

% Dynamically create macros
\newcommand{\makesubfunc}[2]{
	\ifcsname#1\endcsname
		% Macro exists, append to end
		\expandafter\edef\csname#1\endcsname{\expandafter\csname#1\endcsname #2}
	\else
		% Macro doesnt exist, create a new one
		\expandafter\def\csname#1\endcsname{#2}
		% add to list of created macros
		\ifcsname subfunctions\endcsname
			\edef\subfunctions{\subfunctions, #1}
		\else
			%subfunctions macro doesnt exist yet
			\def\subfunctions{#1}
		\fi
	\fi
}

\newcounter{topfunc}
\newcounter{topconc}[topfunc]

\begin{forest}
	for tree={
		fill=white,
		grow'=0,
		parent anchor=children,
		child anchor=parent,
%	  	align=center, 
		text width=4cm,
		draw,
		anchor=parent,
		if n children=0{folder}{},
		l sep=1cm,
		edge path'={[-Latex,line width=0.7pt](!u.parent anchor) -- ++(5pt,0) |- (.child anchor)},
		fit=band,
	},
	where n=1{
		calign with current edge
	}{},
	set root/.style={
		text width=0cm,
		draw=none,
		fill=none
	},
	set function/.style={
		content/.wrap 2 pgfmath args={\textbf{F##1} ##2}{n()}{content()},
		name/.wrap pgfmath arg={f##1}{n()},
		s sep=1cm,
		tree node
	},
	set concept/.style={
		content/.wrap 3 pgfmath args={\textbf{C##1\alphalph{##2}} ##3}{n("!u")}{n()}{content()},
		name/.wrap 2 pgfmath args={f##1c##2}{n("!u")}{n()},
		tree node
	},
	set subfunction/.style n args={3}{
		content/.wrap 2 pgfmath args={\textbf{sF##1} ##2}{n()}{content()},
		name/.wrap 3 pgfmath args={f##1c##2sf##3}{n("!uu")}{n("!u")}{n()},
		TeX={
			\makesubfunc{f\alphalph{#1}c\alphalph{#2}}{(f#1c#2sf#3)}
		},
		tree node
	},
	set subconcept/.style n args={4}{
		content/.wrap 3 pgfmath args={\textbf{sC##1\alphalph{##2}} ##3}{n("!u")}{n()}{content()},
		name/.wrap 4 pgfmath args={f##1c##2sf##3sc##4}{n("!uuu")}{n("!uu")}{n("!u")}{n()},
		TeX={
			\makesubfunc{f\alphalph{#1}c\alphalph{#2}}{(f#1c#2sf#3sc#4)}
		},
		tree node,
	},
	my label/.style={
 	  	label={[align=left,text width=4cm,yshift=0.1cm] {\textbf{#1}}},
	},
	my sflabel/.style n args={2}{
		label={[align=left,text width=4cm,yshift=0.1cm] {\textbf{Sub-Functions (C#1\alphalph{#2})}}},
	},
	before typesetting nodes={
		for tree={
			if level=0{
				% Don't show the root node
				content={},
				set root,
				s sep=1cm,
			}{
				if level=1{
					% Dont connect function nodes to root
					no edge,
					set function,
					TeX={
						\stepcounter{topfunc}
						\edef\toplevel{\toplevel (f\arabic{topfunc})}
					},
					if n=1{
						% Add the "Top Level" label to the first function node
						my label={Top Level},
						%name/.wrap 4 pgfmath args={f##1c##2sf##3sc##4}{n("!uuu")}{n("!uu")}{n("!u")}{n()},
%						TeX={
%							\edef\toplevel{\toplevel (toplevellabel)}
%						},
					}{}
				}{
					if level=2{
						set concept,
						TeX={
							\stepcounter{topconc}
							\edef\toplevel{\toplevel (f\arabic{topfunc}c\arabic{topconc})}
						}
					}{
						if level=3{
							set subfunction/.process={PPP}{n("!uu")}{n("!u")}{n()},
							if n=1{
								my sflabel/.process={PP}{n("!uu")}{n("!u")},
							}{},
						}{
							if level=4{
								set subconcept/.process={PPPP}{n("!uuu")}{n("!uu")}{n("!u")}{n()},
							}{
							},
						},
					},
				},
			},
		},
	},
	[
		[Accelerate relative to Ground
			[Motor and Wheels
				[Transfer energy from motor to wheels
					[Direct drive]
					[Gear box]
					[Belt drive]
					[Chain drive]
					[Friction drive]
					[Fan or propellor]
				]
			]
			[Motor and treads
				[Transfer energy from motor to treads
					[Direct drive with belt]
				]
			]
			[Explosion and wheels
				[Move explosives behind center of mass,
					[Chute of gunpowder with a servo]
					[Gravity fed tube]
					[Small fan]
				]
				[Set Off Explosives
					[Lighter]
					[Primer with pins]
					[Arcing electricity]
				]
			]
			[Jet Engine
				[Hold Fuel
					[Internal gas tank]
					[External gas tank]
				]
				[Move fuel to engine
					[Gravity]
					[Injection]
					[Difference in air pressure]
				]
				[Expel fuel
					[Fan or propellor]
					[Air compressor]
				]
				[Light fuel
					[Pilot light]
					[Arcing electricity]
				]
			]
		]
		[Decelerate relative to ground
			[Drum
				[Apply pressure to wheels
					[Servo]
					[Hydraulics]
					[Pneumatics]
				]
			]
			[Spike
				[Force spike into ground
					[Servo motor]
					[Heavy spike with release latch]
					[Shoot spike into ground]
					[Hydraulics]
					[Pneumatics]
				]
			]
			[Disk brakes
				[Apply pressure to wheels
					[Servo]
					[Hydraulics]
					[Pneumatics]
					[Electromagnets]
				]
			]
			[Drag based brakes
				[Increase drag coefficient
					[Flip panels outward]
					[Parachute]
				]
			]
			[Regenerative brake system
				[Change  from regular motor to generator
					[Servo motor with gears on an axle]
					[Buying a motor that does it automatically]
				]
			]
		]
		[Detect turns
			[Potentiometer]
			[Accelerometer/gyrometer]
			[Light sensor]
			[Preset time intervals]
			[None
				[Go slow enough to not need turn detection]
			]
		][Turn relative to ground
			[Conical wheels
				[Adjust speed for each wheel to acount for difference in distance travelled
					[Geometry will account]
				]
			]
			[Flat wheels
				[Allow for slip
					[Use material with low coefficient of friction]
				]
			]
			[Slip differential]
		]
		[Move up a hill
			[High momentum
				[Quickly accelerate
					[Lightweight chassis]
					[Gear ratio]
				]
			]
			[High torque
				[Increased amounts of friction
					[Rubberized wheels]
					[Adhesive]
				]
				[Sustain torque
					[Gear ratio]
					[Multiple motors]
				]
			]
			[Sticky arms
				[Move along its length
					[Ratchet]
					[Pulley and motor powered]
				]
				[Launch arms forward
					[Crossbow]
					[Pneumatics]
					[Explosives]
					[Slingshot]
					[Catapult]
				]
			]
		]
		[Contain components
			[Sit in box]
			[Drill to plate]
			[Adhere to plate]
			[Potting]
			[Welding]
		]
		[Connect to cargo carts
			[Screws]
			[Rope/wire]
			[Links]
			[Pins]
			[Glue]
		]
		[Stay aligned with rails
			[Conical wheels]
			[Wheels with o-rings and flanges]
			[Tank treads]
		]
	]
	\draw [-Latex,line width=0.7pt] (f3) edge (f4);
	\begin{scope}[on background layer]
		\node [draw=black!60,fill=black!20, fit={\toplevel}] {};
		\ForEachX{,}{
			\node [draw=black!60,fill=black!20, fit={\csname\thislevelitem\endcsname}] {};
		}{\subfunctions}
	\end{scope}
\end{forest}
\end{document}

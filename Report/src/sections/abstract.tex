\documentclass[class=../report, crop=false]{standalone}

\begin{document}

\begin{abstract}
	The objective of the Rail-Rider competition is to design a battery-powered locomotive prototype capable of safely transporting cargo around a track for Lectro-Rail.
	Lectro-Rail commissioned a prototype design of an autonomous, low-cost, and aesthetic locomotive for future applications in railway transportation.
	The locomotive must complete 4 courses with steep hills and tight corners while carrying the most cargo in the fastest time possible.


	Our team’s objective was to design a locomotive that could complete all the courses without derailing.
	We decided to neglect speed and focus on reliability in hill climbs and cornering situations.
	We began this approach by using function decomposition to identify the most important purposes of the locomotive.
	After generating conceptual solutions to these locomotive functions, we used a process of screening to eliminate ideas that violated regulations or were unachievable with our resources.
	We generated multiple physical and analytical prototypes to gather information on the remaining ideas.
	We selected the best concept by emphasizing prototype testing in combination with ranking and scoring.


	The physical prototypes and simulations consistently demonstrated that a dual motor gear transmission was the best concept for the competition.
	We implemented this design alongside an electronic braking system which would allow us to travel faster along straightaways.
	However, this design performed poorly in the competition; our train failed to move due to loose connections in the circuitry.
	The design was also incapable of producing enough torque to climb hills or haul cargo.
	Poor implementation and planning of our design resulted in a 17th place finish.

	We recommend a more reliable mechanical design based on the design of Team D3. 
	\begin{itemize}
		\item Use a higher gear ratio to complete the track and haul cargo
		\item Use a heavier chassis to increase traction force
		\item Use a longer chassis to increase space for electrical and mechanical components
		\item Use an articulating joint to increase stability in corners
	\end{itemize}
	  
	We believe our electronic braking system is promising; it is low cost compared to other forms of speed control and it allows for faster speeds on straight tracks.
	However, due to the time constraints in this competition, prioritizing a reliable mechanical design is more important to the prototype’s success.
\end{abstract}


\end{document}

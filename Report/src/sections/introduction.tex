\documentclass[class=../report, crop=false]{standalone}

\begin{document}

\section{Introduction}

Over a quarter of Canada’s greenhouse gas emissions come from transportation; freight accounts for nearly half of all transportation.
Electric trains are one of the most energy efficient and economical methods of transporting freight; however, they require conductive rails to operate (Ostafichuk, d’Entremont, Fengler, 2018).
Our client, Lectro-Rail, suggests that a battery operated train could circumvent this issue and allow the trains to travel on already developed tracks.
Additionally, Lectro-Rail requested that the train be capable of traveling across diverse topography, controlling speed, hauling cargo, and ascending steep hills.

Our team designed a G-Scale locomotive prototype to compete against 19 other teams in the following rounds:

\begin{enumerate}
	\item A 45$^{\circ}$ hill climb
	\item A timed course over flat track
	\item A timed course while hauling cargo uphill
	\item A timed course over dangerous track 
	\item A division relay race
\end{enumerate}

Our team created a functional decomposition* diagram, Pugh chart*, and Weighted Decision-Making Matrix*(WDM) to aid in the generation of our final design.
We compared our final prototype to others at the competition and chose the best aspects of each to recommend for future use if this competition were to occur again.
The sections of this report reflect the design process we followed:

\begin{itemize}
	\item Strategy
	\item Functional Decomposition
	\item Conceptual Solutions
	\item Evaluation
	\item Final Prototype and Competition Results
	\item Recommendations
	\item Conclusion
\end{itemize}

\end{document}

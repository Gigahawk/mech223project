\documentclass[class=../report, crop=false]{standalone}

\begin{document}

\section{Strategy}

We created a strategy that was derived from the specifications provided by the Rail-Rider competition.
We considered the scoring system, physical attributes of the track, and our own expectations to create the design objectives for our train.
Our resulting strategy, coupled with a formal design process, established a strong foundation for our team to begin the development of our locomotive prototype.

We analyzed the scoring system of the competition in order to generate an appropriate strategy.
We prioritized reliable completion of the track in our selection process because derailment penalties and track completion points are present in each round.
Since many courses require the locomotive to climb hills, we made high torque a priority in our design.
Similarly, we prioritized cornering capability due to the large number of turns in many rounds.
We accepted the resulting tradeoffs in other scoring criteria such as speed and cost.
These criteria were consistent with our expectations of creating a reliable locomotive.

We followed a simplified design process to achieve our competition goals.
To optimize our efficiency, we divided our team into sub teams (Electrical, Gear train, and Testing).
We spent less time in concept generation and preliminary evaluation due to the strict time constraint on the project.
This decision focused our energy into quickly identifying the most promising concept.
We developed a prototype early and emphasized testing to provide the detailed information required for scoring.
The extra time allocated to designing the final prototype provided the opportunity to iterate until a satisfactory design was conceived.


\end{document}
